%%%%%%% By Michał Swoboda
%%%%%%% Poprawki by Błażej Kowalczyk
%%%%%%% Uporządkował pewne rzeczy Bartłomiej Kurosz
%%%%%%% Więcej porządków i poprawek Gabriela Kaczmarek

\usepackage{lmodern}
\usepackage{polski}
\usepackage[utf8]{inputenc}
\usepackage{amsfonts}
\usepackage{tikz}
\usepackage{wrapfig}
\usepackage{pdfpages}
\usepackage{dirtree}
\usepackage{booktabs}
\usepackage{csvsimple}

\usepackage{listings}   
\usepackage{color}
\definecolor{dkgreen}{rgb}{0,0.6,0}
\definecolor{gray}{rgb}{0.5,0.5,0.5}
\definecolor{mauve}{rgb}{0.58,0,0.82}
\lstset{
	language=C,                % the language of the code
	basicstyle=\small,           % the size of the fonts that are used for the code
	numbers=left,                   % where to put the line-numbers
	numberstyle=\footnotesize\color{gray},  % the style that is used for the line-numbers
	stepnumber=1,                   % the step between two line-numbers. If it's 1, each line 
	% will be numbered
	numbersep=5pt,                  % how far the line-numbers are from the code
	backgroundcolor=\color{white},      % choose the background color. You must add \usepackage{color}
	showspaces=false,               % show spaces adding particular underscores
	showstringspaces=false,         % underline spaces within strings
	showtabs=false,                 % show tabs within strings adding particular underscores
	%frame=single,                   % adds a frame around the code
	rulecolor=\color{black},        % if not set, the frame-color may be changed on line-breaks within not-black text (e.g. comments (green here))
	tabsize=2,                      % sets default tabsize to 2 spaces
	captionpos=b,                   % sets the caption-position to bottom
	breaklines=true,                % sets automatic line breaking
	breakatwhitespace=false,        % sets if automatic breaks should only happen at whitespace
	%title=\lstname,                   % show the filename of files included with \lstinputlisting;
	% also try caption instead of title
	keywordstyle=\color{blue},          % keyword style
	commentstyle=\color{dkgreen},       % comment style
	stringstyle=\color{mauve},         % string literal style
	escapeinside={\%*}{*)},            % if you want to add LaTeX within your code
	morekeywords={*,...},              % if you want to add more keywords to the set
	deletekeywords={...}              % if you want to delete keywords from the given language
}

%polish signs in lst code
\lstset{literate=%
	{ą}{{\k{a}}}1
	{ć}{{\'c}}1
	{ę}{{\k{e}}}1
	{ł}{{\l}}1
	{ń}{{\'n}}1
	{ó}{{\'o}}1
	{ś}{{\'s}}1
	{ż}{{\.z}}1
	{ź}{{\'z}}1
	{Ą}{{\k{A}}}1
	{Ć}{{\'C}}1
	{Ę}{{\k{E}}}1
	{Ł}{{\L}}1
	{Ń}{{\'N}}1
	{Ó}{{\'O}}1
	{Ś}{{\'S}}1
	{Ż}{{\.Z}}1
	{Ź}{{\'Z}}1
}

%%%%%%%%%%%%%%%% Początek poleceń własnych
% Maksymalna dostępna wysokość pola w~prezentacji (między wąskim nagłówkiem a~stopką)
% dobrana eksperymentalnie - może w~przyszłości po prostu ją wyliczać???
\newlength{\maxheight}
\setlength{\maxheight}{\paperheight}
\addtolength{\maxheight}{-17.85pt}  % tyle zajmuje naczółek ze stopką 

% Polecenie do dokładania wycentrowanych rzeczy (zdjęć) na środku slajdu 
% bez tytulariów (ale ze stopką i~naczółkiem)
% By uzyskać obrazek "na całą stronę" jako argumentu należy użyć
% \includegraphics[height=\maxheight,width=\paperwidth]{figure/balbina.jpg}
% Jeśli nie chcesz zmian proporcji - zrezygnuj z~jednego z~wymiarów
% Obrazki za duże przykryją naczółek strony
% By wszystko było na swoim miejscu potrzebna jest dwukrotna kompilacja 
\newcommand{\framecentered}[1]{
  \setbeamertemplate{background canvas}{}
  \begin{frame}[c]
    \begin{tikzpicture}[overlay, remember picture]
      \node[anchor=center] at (current page.center) 
      {#1};
    \end{tikzpicture}
  \end{frame}}
%%%%%%%%%%%%%%%% Koniec poleceń własnych

%%%%%%%%%%%%%%%%Początek ustawień
\usebackgroundtemplate{%
  \includegraphics[width=\paperwidth,height=\paperheight]{background/tlo_bezlogo.pdf}} 

\usepackage{beamerthemesplit}
\useoutertheme{infolines}
\useinnertheme{rounded}

\definecolor{konar2}{RGB}{240,152,52}
\definecolor{konar}{RGB}{151,58,66}

\setbeamercolor{block title}{fg=black,bg=konar2}
\setbeamercolor{block title alerted}{fg=konar2,bg=black}

\setbeamertemplate{navigation symbols}{}
\setbeamercolor{frametitle}{fg=white,bg=konar}
\setbeamercolor{section in head/foot}{bg=konar}
\setbeamercolor{author in head/foot}{fg=Black,bg=konar2}
\setbeamercolor{date in head/foot}{fg=Black}
\setbeamercolor{title in head/foot}{fg=white, bg=konar}
\setbeamercolor{section in head/foot}{fg=white}
\setbeamercolor{titlelike}{fg=black}
\setbeamercolor{structure}{bg=black, fg=konar2}
\setbeamercolor{subsection in head/foot}{fg=black}
\setbeamercolor{item}{bg=white}

%%%%%%%%%%%%%%%%%Koniec Ustawień

